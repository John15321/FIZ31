\documentclass{article}
\usepackage[utf8]{inputenc}
\usepackage{array}
\usepackage{wrapfig}
\usepackage{multirow}
\usepackage{tabularx}
\usepackage{amsmath}
\usepackage{mathtools}
\usepackage[table]{xcolor}
\title{Sprawozdanie 1}

\author{Jan Bronicki\\
Nr. indeksu: 249011}
\date{}
\begin{document}

\maketitle


%---------------------------------------------------------------------

\begin{center}
    \renewcommand{\arraystretch}{1.2}
\begin{tabular}{ |c|c|c|c|c|c|c|c|c|c| }
    \hline
    U(V)&u(U)&I[mA]&R[$\Omega$]&$u_c(R)$&$\bar{R}$&$u(\bar{R})$&$R_w[\Omega]$&$u_c(R_w)$ \\
    \hline \hline
    3.29&0.02&18.7&0.19&&&&&\\ 
    \hline
    4.78&0.02&27.8&0.25&&&&&\\ 
    \hline
    6.35&0.02&36.1&0.31&&&&&\\ 
    \hline
    7.89&0.03&44.9&0.37&&&&&\\ 
    \hline
    9.50&0.03&54.2&0.43&&&&&\\ 
    \hline
    12.44&0.04&71.0&0.55&&&&&\\ 
    \hline
\end{tabular}
\end{center}
%---------------------------------------------------------------------
Przykładowe obliczenia:
\begin{center}
    $\Delta u(U)  = \pm0.5\% \cdot rdg+1 \cdot dgt=\cfrac{0.5}{100} \cdot 3.29 + 0.02=0.0264\approx0.03$\\
    $u(U)=\cfrac{\Delta u(U)}{\sqrt{3}}=0.015\approx0.02$
\end{center}

\end{document}
