
\documentclass{article}
\usepackage[utf8]{inputenc}
\usepackage{subcaption}
\usepackage{graphicx}
\usepackage[margin=2.5cm]{geometry}
\usepackage{array}
\usepackage{wrapfig}
\usepackage[utf8]{inputenc}
\usepackage[english]{babel}

\usepackage{multicol}
\usepackage{multirow}
\usepackage{tabularx}
\usepackage{amsmath}
\usepackage{wrapfig}
\usepackage{mathtools}
\usepackage[table]{xcolor}
\usepackage{xcolor,colortbl}
\usepackage{multirow}
\usepackage{polski}
\title{Sprawozdanie 2}

\author{
Ćwiczenie: 8}
\date{}
\begin{document}

\maketitle
\begin{multicols}{3}
    \begin{table}[h]
       \centering
       \begin{tabular}{c|c}
           aaa & aaa \\
           bbb & bbb
       \end{tabular}
       \caption{Caption}
       \label{tab:my_label}
    \end{table}
    % \columnbreak %Column break point
    \begin{table}[h]
       \centering
       \begin{tabular}{c|c}
           aaa & aaa \\
           bbb & bbb
       \end{tabular}
       \caption{Caption}
       \label{tab:my_label}
    \end{table}
    % \columnbreak %Column break point
    \begin{table}[h]
       \centering
       \begin{tabular}{c|c|c|c|c|}
           aaa & aaa & aaa & aaa & aaa \\
           bbb & bbb & aaa & aaa & aaa 
       \end{tabular}
       \caption{Caption}
       \label{tab:my_label}
    \end{table}
    \end{multicols}

\end{document}