\documentclass{article}
\usepackage[utf8]{inputenc}
\usepackage{subcaption}
\usepackage{graphicx}
\usepackage[margin=2.5cm]{geometry}
\usepackage{array}
\usepackage{wrapfig}
\usepackage[utf8]{inputenc}
\usepackage[english]{babel}

\usepackage{multicol}
\usepackage{multirow}
\usepackage{tabularx}
\usepackage{amsmath}
\usepackage{wrapfig}
\usepackage{mathtools}
\usepackage[table]{xcolor}
\usepackage{xcolor,colortbl}
\usepackage{multirow}
\usepackage{polski}
\title{Sprawozdanie 2}

\author{
Ćwiczenie: 8}
\date{}
\begin{document}

\maketitle
%------------------------------------------------------------------
% WSTEP TEORETYCZNY
\section{Wstęp Teoretyczny}
\par Pomiar współczynnika lepkości $\eta$ cieczy metodą Stokesaza za pomocą 
szerokiego cylindrycznego 
naczynia szklanego. \\
\begin{center}
    $
    \eta=\frac{d^{2}\cdot g\cdot t\cdot (\rho_{k}-\rho_{c})}{18h}
    $
    \begin{flushleft}
        Gdzie:\\
        $d$ - średnica kulki\\
        $g$ - przyspieszenie ziemskie $(9.81\frac{m}{s^{2}})$\\
        $\rho_{k}$ - gęstość kulki\\
        $\rho_{c}$ - gęstość cieszy (gliceryny)\\
        $h$ - długość trasy tonącej w glicerynie kulki
    \end{flushleft}
\end{center}
\par Lepkość zostanie wyznaczona na podstawie danych otrzymanych przez obserwacje kulki 
tonącej w glicerynie. Dzięki analizie ruchu kulki, znając jej parametry takie 
jak masa i średnica, które przekładają się na gęstość. Można zanalizować siły oporu,
które stawia ciecz co przekłada się na współczynnik lepkości $\eta$.\\ \\
W naszym eksperymencie wykorzystamy następujące przyrządy:\\
\begin{itemize}
    \item Naczynie cylindryczne z badaną cieczą (w tym wypadku z gliceryną)
    \item Areometr do zbadania gęstości cieczy
    \item Trzy różne kolorowe kulki (Biała, Czarna i Niebieska)
    \item Waga
    \item Suwmiarka do pomiaru średnicy kulek
    \item Stoper
    \item Linijka z podziałką milimetrową
\end{itemize}
% WSTEP TEORETYCZNY
%------------------------------------------------------------------
\newpage
%------------------------------------------------------------------
% OTRZYMANE POMIARY I ICH OPRACOWANIE
\section{Otrzymane pomiary i ich opracowanie}


% POMIARY KULKI BIALEJ
\begin{center}
    \rowcolors{2}{gray!5}{lightgray!30}
    \begin{table}[h]
        \caption{Wyniki pomiaru kulki Białej}
        \centering
        \begin{tabular}{|l|r|r|r|}
            \hline
            Nr pomiaru & d[m] & m[kg] & t[s] \\ \hline
            1 & 0.008 & 0.000486  & 18.61 \\ \hline
            2 & 0.008 & 0.00048   & 18.48 \\ \hline
            3 & 0.008 & 0.0004824 & 20.36 \\ \hline
            4 & 0.008 & 0.0004844 & 18.18 \\ \hline
            5 & 0.008 & 0.000498  & 18.14 \\ \hline
            6 & 0.008 & 0.0004916 & 18.38 \\ \hline
            7 & 0.008 & 0.0004924 & 18.9  \\ \hline
            8 & 0.008 & 0.0004954 & 18.16 \\ \hline
            9 & 0.008 & 0.0004812 & 18.25 \\ \hline
            10& 0.008 & 0.0004916 & 18.5  \\ \hline
            Srednia: & 0.008 & 0.0004883 & 18.596 \\ \hline
        \end{tabular}%
        \label{tab:Tabela Pomiarow Kulki Bialej}%
    \end{table}%
\end{center}
% POMIARY KULKI BIALEJ

\vspace{-10ex}

% POMIARY KULKI CZARNEJ
\begin{center}   
    \rowcolors{2}{gray!70}{lightgray!80}
    \begin{table}[h]
        \caption{Wyniki pomiaru kulki Czarnej}
        \centering
        \begin{tabular}{|l|r|r|r|}
            \hline
            Nr pomiaru & d[m] & m[kg] & t[s] \\ \hline
            1 & 0.006 & 0.0002364 & 21.83 \\ \hline
            2 & 0.006 & 0.000235  & 22.24 \\ \hline
            3 & 0.006 & 0.0002516 & 21.61 \\ \hline
            4 & 0.006 & 0.0002474 & 21.56 \\ \hline
            5 & 0.006 & 0.0002464 & 21.67 \\ \hline
            6 & 0.006 & 0.0002418 & 21.3  \\ \hline
            7 & 0.006 & 0.0002376 & 21.57 \\ \hline
            8 & 0.006 & 0.0002358 & 21.18 \\ \hline
            9 & 0.006 & 0.0002422 & 22.08 \\ \hline
            10& 0.006 & 0.0002377 & 21.16 \\ \hline
            Srednia: & 0.006 & 0.00024119 & 21.62 \\ \hline
        \end{tabular}%
        \label{tab:Tabela Pomiarow Kulki Czarnej}%
    \end{table}%
\end{center}
% POMIARY KULKI CZARNEJ

% \vspace{-50ex}

% POMIARY KULKI NIEBIESKIEJ
\begin{center}
    \rowcolors{2}{cyan!90}{cyan!40}
    \begin{table}[h]    
        \caption{Wyniki pomiaru kulki Niebieskiej}
        \centering
        \begin{tabular}{|l|r|r|r|}
            \hline
            Nr pomiaru & d[m] & m[kg] & t[s] \\ \hline
            1 & 0.006 & 0.0002364 & 21.83 \\ \hline
            2 & 0.006 & 0.000235  & 22.24 \\ \hline
            3 & 0.006 & 0.0002516 & 21.61 \\ \hline
            4 & 0.006 & 0.0002474 & 21.56 \\ \hline
            5 & 0.006 & 0.0002464 & 21.67 \\ \hline
            6 & 0.006 & 0.0002418 & 21.3  \\ \hline
            7 & 0.006 & 0.0002376 & 21.57 \\ \hline
            8 & 0.006 & 0.0002358 & 21.18 \\ \hline
            9 & 0.006 & 0.0002422 & 22.08 \\ \hline
            10& 0.006 & 0.0002377 & 21.16 \\ \hline
            Srednia: & 0.006 & 0.00024119 & 21.62 \\ \hline
        \end{tabular}%
        \label{tab:Tabela Pomiarow Kulki Niebieskiej}%
    \end{table}%
\end{center}
% POMIARY KULKI NIEBIESKIEJ
\newpage
\begin{multicols}{3}
    \begin{table}[h]
       \centering
       \begin{tabular}{c|c}
           aaa & aaa \\
           bbb & bbb
       \end{tabular}
       \caption{Caption}
       \label{tab:my_label}
    \end{table}
    % \columnbreak %Column break point
    \begin{table}[h]
       \centering
       \begin{tabular}{c|c}
           aaa & aaa \\
           bbb & bbb
       \end{tabular}
       \caption{Caption}
       \label{tab:my_label}
    \end{table}
    % \columnbreak %Column break point
    \begin{table}[h]
       \centering
       \begin{tabular}{c|c|c|c|c|c}
           aaa & aaa & aaa & aaa & aaa & aaa \\
           bbb & bbb & aaa & aaa & aaa & aaa
       \end{tabular}
       \caption{Caption}
       \label{tab:my_label}
    \end{table}
    \end{multicols}

% OTRZYMANE POMIARY I ICH OPRACOWANIE
%------------------------------------------------------------------
\end{document}
