%Sprawozdanie z Fiz3.1 Jan Bronicki i Marcin Radke
\documentclass{article}
\usepackage[utf8]{inputenc}
\usepackage{graphicx}
\graphicspath{ {graph1.png},{graph2.png} }
\usepackage[margin=2.5cm]{geometry}
\usepackage{array}
\usepackage{wrapfig}
\usepackage{multirow}
\usepackage{tabularx}
\usepackage{amsmath}
\usepackage{wrapfig}
\usepackage{mathtools}
\usepackage[table]{xcolor}
\usepackage{xcolor,colortbl}
\usepackage{multirow}
\usepackage{polski}
\definecolor{Gray}{gray}{0.85}
\definecolor{LightCyan}{rgb}{0.88,1,1}  
\title{Sprawozdanie 2}

\author{Jan Bronicki \\
Nr indeksu: 249011\\
Marcin Radke\\
Nr indeksu: 241554\\
Ćwiczenie: 8}
\date{}
\begin{document}

\maketitle
%---------------------------------------------------------------------
\begin{table}[h]
    \begin{flushleft}
        Uzyskane dane oraz ich wyliczone niepewności:\\
    \end{flushleft}
    \renewcommand{\arraystretch}{1.5}
\rowcolors{3}{green!80!yellow!50}{green!70!yellow!40}
\begin{tabular}{ |c|c|c|c|c|c|c|c| }
    \hline
    Lp.&m[kg]&d[m]&h[m]&t[s]&$\rho_k \left[\cfrac{kg}{m^3}\right]$&$\rho_c \left[\cfrac{kg}{m^3}\right]$&$\eta \left[\cfrac{Ns}{m^2}\right]$ \\
    \hline \hline
    3.29&$\pm$0.02&18.7&$\pm$0.2&175.94&$\pm $ 2.16& \multirow{6}{*}{175}&\multirow{6}{*}{$\pm$0.62}\\ 
    \cline{1-6}
    4.78&$\pm$0.02&27.8&$\pm$0.3&171.94&$\pm$ 1.99&&\\ 
    \cline{1-6}
  
    6.35&$\pm$0.02&36.1&$\pm$0.3&175.90&$\pm$ 1.70&&\\ 
    \cline{1-6}

    7.89&$\pm$0.03&44.9&$\pm$0.4&175.72&$\pm$ 1.41&&\\ 
    \cline{1-6}

    9.50&$\pm$0.03&54.2&$\pm$0.4&175.28&$\pm$ 1.51&&\\ 
    \cline{1-6}
    
    12.44&$\pm$0.04&71.0&$\pm$0.6&175.21&$\pm$ 1.58&&\\ 

    \hline

\end{tabular}
\label{tabular: t}
\centering
\end{table}
%------------------------------------------------------------------
\begin{center}
    \begin{flushleft}
        Przykładowe obliczenia:\\
    \end{flushleft}
    Delta niepewności napięcia:\\*
    $\Delta u_p (U)  = 0.5\% \cdot rdg+1 \cdot dgt=$\\
    \vspace{5mm}
    Niepewność napięcia:\\*
    $u_B (U)=\cfrac{\Delta u_p(U)}{\sqrt{3}}=$\\*
    \vspace{5mm}
    Delta niepewności natężenia:\\* 
    $\Delta u(I) = 1.2\% \cdot rdg+1 \cdot dgt =$\\*
    \vspace{5mm}
    Niepewność natężenia:\\*
    $u(I)= \cfrac{\Delta u(I)}{\sqrt{3}} \approx $\\*
    %%%%%%%%%%%%%%%
    \vspace{5mm}
    Niepewność całkowita R:\\*
    $u_c (R)=\sqrt{\sum_{j=1}^{k}\left( \cfrac{\partial f}{\partial x_j}\right)^2 u^2 (x_j)}
    =$\\%%%%%%%%%%%%%%%
    \vspace{5mm}
    Wartość średnia R:\\* 
    $\bar{R} = \cfrac{\sum_{i=1}^{n}x_i}{n}=$\\*
    \vspace{5mm}
    Niepewność wartości średniej R:\\*
    $u(\bar{R}) = \sqrt{\cfrac{\sum_{i=1}^{n} \left(x_i - \bar{x} \right)^2}{n(n-1)}} $\\*



\end{center}
\newpage

\end{document}
